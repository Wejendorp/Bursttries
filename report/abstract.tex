\noindent
{\bf Abstract}
\\\\
With the scaling of CPUs focusing more on increases in the number of cores,
highly parallelisable data structures are of increasing importance.

This project seeks to explore the implementation tradeoffs in parallelising the
burst trie data structure, which has been shown to be on par with hash maps in
both speed and space consumption. The burst trie provides fast access to
strings and integers, however different bucket types have different advantages
and disadvantages.

By comparing different tradeoffs needed to implement partial functionalities,
specialized data structures are developed. Depending on the level of
parallelism, distribution of insertions/searches/removals, and the
distribution of the inserted data, different variations excel. This means
evaluating the modifications needed to make the trie work as a threadsafe set-,
multiset- or map container.

The performance of the trie in sequential testing is found to degrade with
increased bucket sizes, while the opposite tendency is found for parallel
scaling for larger datasets. As such no silver bullet is found for bucket
sizes.

Using a relatively simple locking mechanism, speedups of up to a factor $6$ are
observed for searches and $3.5$ for insertions, when utilizing 8 CPUs.

A lock-free implementation technique is proposed to avoid bottlenecking at the
root of the trie, but not successfully implemented.
\\\\
{\bf Resume}
\\\\
Da skalering af processorer i stigende grad fokuserer på forøgelse af antallet af
kerner, er højt paralleliserbare data strukturer af stigende vigtighed.

Dette projekt søger at undersøge hvilke kompromiser der indgår i implementation
af burst trie datastrukturen, der er vist at være på højde med hash maps
både med ydelse og hukommelses forbrug. Burst trien giver hurtig tilgang til
strenge og heltal, men forskellige bucket typer har forskellige fordele og ulemper.

Ved at sammenligne de forskellige kompromiser nødvendige for at implementere dele
af funktionaliteten udvikles specialiserede datastrukturer. Afhængig af niveauet
af parallelisme, fordeling af indsættelser/søgninger/sletninger, og fordeling
af det indsatte data, vil forskellige variationer være bedst. Dette betyder at
vi må evaluere modifikationerne der er nødvendige for at gøre strukturen til
en trådsikker sæt-, multisæt- og map struktur.

Ydelsen for trien ved sekventiel test ses at gå ned med forøgede bucket størrelse
mens den omvendte tendens ses for parallel skalering for større datasæt. Derfor
findes der ingen gylden bucket stoerrelse.

Med brug af relativt simple locking metoder, observeres speedups på op til en
faktor 6 for søgninger og 3.5 for indsættelser med brug af 8 processorer.

En lock-free implementation foreslås for at undgå flaskehalse ved
roden af trien, men har ikke kunnet implementeres.

