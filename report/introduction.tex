\chapter{Introduction}

\section{Problem definition}
How can the burst trie be parallelised, in a way that allows building and searching
the structure faster than an equivalent serial singlethreaded approach?
What kind of overhead does it create, and does this influence relative
performance between the individual variants of the structure?


\section{Background}
%Define: What is a trie?
A trie is a datastructure for storing keys based on their prefixes.
It is also known as a prefix-tree. The idea is equivalent to that of
"most significant digit" (MSD-)radix sort.

Each level in the tree reads a prefix of the key and branches accordingly.
A trie working on strings with the ASCII alphabet will thus have a branching
factor of 128. Each read character corresponds to a branch and goes down another
level until finding the leaf matching the requested key or terminating
unsuccesfully. A trie is as deep as the longest key that it contains, since
each level removes a character from the key. It is this bound that newer trie
variants attempt to improve, without compromising lookup efficiency.

%Define: What is the idea of a burst trie?
Following the radix sort analogy, the regular trie uses radix-sort until
reaching the trivial case, with no boundary for using other sorting methods.
The burst trie, on the other hand, uses a limit for employing other methods
based on the element count. In other words a burst trie differs from a regular
trie by the fact that it compresses the elements into "bucket" structures when
there are few of them in a particular subtree. Each bucket is then assigned a
capacity. When inserting elements into a burst trie, the structure finds the
corresponding bucket and puts the element in it. If the bucket exceeds its
capacity it "bursts", creating a new parenting node and redistributing the
elements into buckets under it, based on the next part of the prefix. As such,
each bucket is uniquely determined by its prefix (or path), and the elements in
each bucket have the same prefix, making it possible to discard this
information to save space.
This is what makes it possible for the trie to be more space efficient than an
equivalent hashmap, or comparison-based tree structure.
%Weaknesses...


Although the burst trie has been studied with respect to cache optimization, no
work has been done in parallelising the structure in a way that allows
insertions and deletions faster than the sequential implementation, while
remaining space efficient\cite{Askitis2010}. This means evaluating the sacrifices needed to make
the structure threadsafe.

Parallelism in general!!
inspired by ACM .. something.. I want to make lockfree!!



% Why are burst tries nice?
Burst tries have been shown to be faster and more space efficient than
compari\-son-based datastructures such as B-trees and red-black trees
when used on integer data.
For strings the burst trie is more efficient than trees, while unlike
hash-tables allowing efficient iteration of sorted keys with minor modifications.
In that context it would be interesting to see how versatile the structure is,
if it is able to compete with other \textit{(multi-)set} structures and heaps with
relevant modifications.

The original burst trie uses linked lists for the containers, but this has been
shown to be inefficient in a modern cache hierachy. There has been done research
in different ways to improve the structure, by replacing the bucket structure.
Nash and Gregg have done work on optimizing the trie itself, using so-called
level- and path-compression.

In an earlier course, the author has done work on the burst trie as an Integer
datastructure in cooperation with Niels Bjørn Bugge Grathwohl.

% Multithreading!!
While maintaining optimizations for caching remains important, modern processor
scaling primarily focuses on multithreading, making it increasingly important
to create efficient multi-threaded datastructures and algorithms.
To the author's knowledge, no work has been done on parallelising the structure
for modern multi-core processors, as this was left open by Askitis and Sinha. REFERENCE
\fxnote{REFERENCES!}
%//TODO References


