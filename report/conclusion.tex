\chapter{Conclusion}

% Implementing the trie
In extending the structure to support even simple removals, an extra counter
was added to the nodes and the buckets. The implemented removal method
is simplistic to avoid maintaining extra size counts. This has the drawback of
allowing long chains of nodes to remain, when they could be reduced to a bucket.

The operations required by an STL {\keyword (Multi-)map} have been implemented:

For efficient iterators, the buckets are made into a linked list, adding two
pointers per bucket. These are maintained in $O(h+\alpha)$ time in the chosen
implementation.

In order to minimize the linear scanning, {\keyword min} and {\keyword max}
indices are added to the nodes. These are maintained in $O(\alpha)$, by
scanning the children array. The more efficient solution of bitvectors has not
been implemented, which would allow this to be reduced to constant work.

The linked list of buckets is also used for finding the predecessor and
successor to a given element, scanning the bucket. If no bucket is found
linear scanning of the children array is performed, until one is found.
As such these methods come for free when the iterators have been implemented.

The differences between a set and a map is only one of inserting pointers together
with the keys. As such no changes are required. The only difference between a
map and a multi-map is made in the buckets. If a multi-map is desired, the
buckets need to insert a new key regardless of preexisting keys. If a map/set is
wanted, the buckets need to determine if the key already exists.
The implemented structure does not include this check, and is therefore
a multi-map.
\\

% Parallelising the trie
Using a relatively simple locking mechanism, speedups of up to a factor $6$
were observed for searches and $3.5$ for insertions, when utilizing 8
processors.

In theory the read-locks allow full concurrency, and as such linear scaling to the
processor count was expected. The results show that the one lock per node
approach is not a globally suited solution, since the scaling is very dependent
on the height of the trie.

The exclusive locking comes at the cost of negative speedup for over-saturated
processors due to synchronization, and when combined with very efficient bucket
operations becomes the dominant factor for larger tries.

The waiting itself is the problem for smaller tries, since the locking is
incremental from the root down, creating what is essentially a sequential
locking queue at the root. In order to obtain increased parallelism, smaller
bucket sizes will increase the size of the trie.

\section{Future work}
A caveat of the chosen implementation is the lack of a reduction criteria
for removals beyond the naive. Such a criteria will be able to guarantee
that a reduction is only one level at a time. This is expected to  make
reasoning about a lock-free removal solution easier.

The implementation of a lock-free trie was unsuccessful, and remains a
theoretical possibility. The theoretical ground work has been laid for
implementing a wait-free trie, both with and without assisted bursting.

If such a structure is implemented, the perceived locking granularity of each
node will be increased by a factor of $\alpha$, as a result of backing-off on
concurrent changes of the same memory location. No locking actually occurs,
and the resulting structure can scale to any number of processors.
