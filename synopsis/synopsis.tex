\documentclass[a4paper, oneside, draft]{memoir}
\setcounter{secnumdepth}{-1} % Sæt overskriftsnummereringsdybde. Disable = -1.
\newcommand{\projecttitle}{Implementing and parallelising Bursttries}
\newcommand{\bigname}[1]{
        \begin{center}\fontfamily{phv}\selectfont\large\scshape#1\end{center}
}
\title{Project synopsis\\\projecttitle}
\author{Jacob Wejendorp}
\renewcommand{\bibsection}{\section*{References}}
\begin{document}
\maketitle
%Project written under the advisory of Amr Elmasry.\\\\


\clearpage
%Formål: Synopen er en projektbeskrivelse, som defierer opgaven. Synopsen bliver dermed læst af censor.
%Synopsen kan ændres undervejs i projektet efter aftale med vejleder (husk at flevere den ændrede version)


\bigname{\projecttitle}

\section{Problem definition}
How can the burst trie be parallelised, in a way that allows building and searching
the structure faster than an equivalent serial singlethreaded approach?
What kind of overhead does it create, and does this influence relative
performance between the individual variants of the structure?

%Rainbow tables ... is not an option.
 %How can the burst trie datastructure be adapted to bit-strings and char-strings
%(fixed-width characters), and what kind of time- resp. memory-bounds can each obtain?
%Are these bounds on par with other structures used for maps, sets or heaps, and
%how is the practical performance?
%\\or\\
%I wish to verify the  

%Problemformulering
\section{Introduction}
%Working with rainbow tables ... is not an option.
Working with large datasets is becoming increasingly relevant, and our
datasets are growing faster than our scaling in cpu-power. This makes optimizing
these datasets still relevant.

\subsection{What is a trie?}
%Define: What is a trie?
A trie is a datastructure for storing keys based on their prefixes.
It is also known as a prefix-tree. The idea is equivalent to that of
"most significant digit" (MSD-)radix sort.

Each level in the tree reads a prefix of the key and branches accordingly.
A trie working on strings with the latin alphabet will thus have a branching
factor of 26. Each read character corresponds to a branch and goes down another
level until finding the leaf matching the requested key or terminating
unsuccesfully. A trie is as deep as the longest key that it contains, since
each level removes a digit from the key. It is this bound that newer trie
variants attempt to improve, without compromising lookup efficiency.

\subsection{Burst tries}
%Define: What is the idea of a burst trie?
Following the radix sort analogy, the regular trie uses radix-sort until
reaching the trivial case, with no boundary for using other sorting methods.
The burst trie, on the other hand, uses a limit for employing other methods
based on 
the element count. In other words a burst trie differs from a regular trie by
the fact that it compresses the leaves into "bucket" structures when there are
few of them in a particular subtree. Each bucket is then assigned a capacity.
When inserting elements into a burst trie, the structure finds the corresponding
bucket and puts the element in it. If the bucket exceeds its capacity it 
"bursts", creating a new parenting node and redistributing the elements 
into buckets under it, based on the next part of the prefix.
As such, each bucket is uniquely determined by its prefix (or path),
and the elements in each bucket have the same prefix, making it possible to
discard this information to save space.

\section{Motivation}
% Why are burst tries nice?
Burst tries have been shown to be faster and more space efficient than
compari\-son-based datastructures such as B-trees and red-black trees
when used on integer data.
For strings the burst trie is more efficient than trees, while unlike
hash-tables allowing efficient iteration of sorted keys with minor modifications.
In that context it would be interesting to see how versatile the structure is,
if it is able to compete with other \textit{(multi-)set} structures and heaps with
relevant modifications.

The original burst trie uses linked lists for the containers, but this has been
shown to be inefficient in a modern cache hierachy. There has been done research
in different ways to improve the structure, by replacing the bucket structure.
Nash and Gregg have done work on optimizing the trie itself, using so-called
level- and path-compression.

In an earlier course, the author has done work on the burst trie as an Integer
datastructure in cooperation with Niels Bjørn Bugge Grathwohl.

% Multithreading!!
While maintaining optimizations for caching remains important, modern processor
scaling primarily focuses on multithreading, making it increasingly important
to create efficient multi-threaded datastructures and algorithms.
To the author's knowledge, no work has been done on parallelising the structure
for modern multi-core processors, as this was left open by Askitis and Sinha.


%Another interesting approach could be implementing an interface for exploiting 
%the bit-string approach for other datatypes. This could yield an aggregate structure,
%using fixed-width characters for the first levels and bit-strings for the next.

\section{Learning objectives}
The project will make the student able to:
\begin{enumerate}
    \item Read and implement datastructures based on leading research papers.
    \item Implement and reason about parallel algorithms.
\end{enumerate}

%Begrundelse (hvorfor er emnet interessant dette er
%essentielt en uddybning af problemformuleringen)

\section{Tasks and Time schedule}
Time schedule is based on work in the weeks 6 to 23.
%Arbejdsopgaver og tidsplan
\begin{enumerate}
    \item Preliminary research (1-2 weeks):
        \begin{itemize}
        \item Find related research papers, resulting in a preliminary bibliography.
        \item Write up the basic burst trie algorithm.
        \item Write up alterations based on related work.
        \end{itemize}
    \item Theory (3 weeks):
        \begin{itemize}
        \item Write proofs for the bounds for the basic structure, and for proposed
          changes/tradeoffs.
        \item Investigate possible tradeoffs for static tries, and for map-, set-
            and multi-set functionality of tries.
        \item Investigate possible locking mechanism to allow multithreaded
            processing and updating.
        \end{itemize}
    \item Implementation (4 weeks):
        \begin{itemize}
        \item Write C++ implementation of Burst trie variants based on design
                documents from step 1.
        \item Modify implementations to allow multithreaded processing.
        \end{itemize}
    \item Testing (2 weeks):
        \begin{itemize}
        \item Write a test-suite for the datastructures.
        \item Perform tests using different datasets, with 
                skewed/even distributions, different key lengths and data sizes.
        \end{itemize}
    \item Write report (3 weeks)
    \item (2 weeks) Buffer for possible complications and expansions.
\end{enumerate}

\section{Possible extensions}
\begin{enumerate}
    \item Investigate tries as a cache-oblivious structure, eliminating the
        demand for staying within RAM-limits.
\end{enumerate}
\nocite{Badr:2005, Dorji:2010, Nash2008, wejendorpgrathwohl2010, Heinz2002}
\bibliographystyle{plain}
\bibliography{bibliography}

\end{document}
