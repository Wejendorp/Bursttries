%\documentclass{DIKU-article}
\documentclass{article}

\usepackage[utf8]{inputenc}
\usepackage[english]{babel}
% \usepackage{fullpage}
\usepackage{amsmath}
\usepackage{amssymb}
\usepackage{latexsym}
\usepackage[pdftex,colorlinks]{hyperref}
% \hypersetup{colorlinks=false}
% \usepackage[labelfont=bf,up,textfont=it,up]{caption}
\usepackage{graphicx}
%\usepackage{subfigure}
%\usepackage{multirow}
%\usepackage{verbatim}
%\usepackage{listings}
\newcommand{\projecttitle}{Exploring Bursttries}
\newcommand{\bigname}[1]{
        \begin{center}\fontfamily{phv}\selectfont\large\scshape#1\end{center}
}
\title{Bachelorproject\\\projecttitle}
\author{Jacob Wejendorp}

\begin{document}
\maketitle
Project written under the advisory of Amr Elmasry.\\\\

Meetings when needed?\\
Correspondance via E-mail prior to meetings
\newpage

%Formål: Synopen er en projektbeskrivelse, som defierer opgaven. Synopsen bliver dermed læst af censor.
%Synopsen kan ændres undervejs i projektet efter aftale med vejleder (husk at flevere den ændrede version).


\bigname{\projecttitle}

\section*{Problem definition}
How can the burst trie datastructure be adapted to bit-strings and char-strings
(fixed-width characters), and what kind of time- resp. memory-bounds can each obtain?
Are these bounds on par with other structures used for maps, sets or heaps, and
how is the practical performance?
\\or\\
I would like to 

%Problemformulering


\section*{Motivation}
Define: What is a trie?
Define: What is the idea of a burst trie?


Burst tries have been shown to be faster and more space efficient than
comparison-based datastructures such as B-trees and red-black trees
when used on integer data\cite{Nash2008}, and for strings more efficient than trees,
while unlike hash-tables allowing efficient iteration of sorted keys with
minor modifications.\cite{wejendorpgrathwohl2010}
In that context it would be interesting to see how versatile the structure is,
if it is able to compete with other {\tt (multi-)set} structures and heaps with
relevant modifications.

In extension of prior work in \cite{wejendorpgrathwohl2010}, the trie's properties
with fixed-width characters instead of bit-strings is a logical next step.

Another interesting approach could be implementing an interface for exploiting 
the bit-string approach for other datatypes. This could yield an aggregate structure,
using fixed-width characters for the first levels and bit-strings for the next.


%Begrundelse (hvorfor er emnet interessant dette er
%essentielt en uddybning af problemformuleringen)

\section*{Tasks and Time schedule}
%Arbejdsopgaver og tidsplan
\begin{enumerate}
    \item Find related research papers, resulting in a preliminary bibliography.
    \item Write up exact datastructure algorithms for insertion/deletion/search.
          Works as a design document for step 3. Details to be found in \cite{Heinz2002}.
    \item Write a C++ implementation of Burst tries based on 2, as claimed best in \cite{Nash2008}.
    \item Perform practical tests with md5 rainbow tables, comparing base performance
          with c++ stl map and set structures.
          
    \item Write proofs for the bounds for the basic structure, and for proposed
          changes/tradeoffs. That is, set-, map-, bit-string- and char-string-
          specific optimizations.
    \item Implement and test the changed burst tries on the md5 tables.
\end{enumerate}


\section*{Methodology and literature}
%Evt. metodiske overvejelser, relevant litteratur o.lign.

I would like to verify the theoretical bounds claimed by \cite{Nash2008}.
?
\nocite{Badr:2005, Dorji:2010}

\bibliographystyle{abbrv}
\bibliography{bibliography}

\end{document}
